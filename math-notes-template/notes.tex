\documentclass[oneside]{book}

\usepackage{amsmath, amsthm, amssymb, amsfonts}
\usepackage{thmtools}
\usepackage{graphicx}
\usepackage{setspace}
\usepackage{geometry}
\usepackage{float}
\usepackage{hyperref}
\usepackage[utf8]{inputenc}
\usepackage[english]{babel}
\usepackage{framed}
\usepackage[dvipsnames]{xcolor}
\usepackage{environ}
\usepackage{tcolorbox}
\tcbuselibrary{theorems,skins,breakable}
\usepackage{eurosym}
\usepackage{enumerate}

\setstretch{1.2}
\geometry{
    textheight=9in,
    textwidth=5.5in,
    top=1in,
    headheight=12pt,
    headsep=25pt,
    footskip=30pt
}

% Variables
\def\notetitle{MATH 423}
\def\noteauthor{
    \textbf{Omer Junedi} \\
    University of Michigan
    }
\def\notedate{Winter 2025}

% The theorem system and user-defined commands
\input{theorems.tex}
\input{commands.tex}
\newcommand{\w}{\omega}

% ------------------------------------------------------------------------------

\begin{document}
\title{\textbf{
    \LARGE{\notetitle} \vspace*{10\baselineskip}}
    }
\author{\noteauthor}
\date{\notedate}

\maketitle
\newpage

\tableofcontents
\newpage

% ------------------------------------------------------------------------------

\chapter{Examples}

\section{Theorem System}

\defn{Definition Name}{
    A defintion.
}

\thmr{Theorem Name}{mybigthm}{
    A theorem.
}

\lem{Lemma Name}{
    A lemma.
}

\fact{
    A fact.
}

\cor{
    A corollary.
}

\prop{
    A proposition.
}

\clmp{}{
    A claim.
}{
    A reference to Theorem~\ref{thm:mybigthm}
}

\pf{
    Veniam velit incididunt deserunt est proident consectetur non velit ipsum voluptate nulla quis. Ea ullamco consequat non ad amet cupidatat cupidatat aliquip tempor sint ea nisi elit dolore dolore.

    Laboris labore magna dolore eiusmod ea ex et eiusmod laboris. Et aliquip cupidatat reprehenderit id officia pariatur.
}

\ex{
    Nostrud esse occaecat Lorem dolore laborum exercitation adipisicing eu sint sunt et. Excepteur voluptate consectetur qui ex amet esse sunt ut nostrud qui proident non. Ipsum nostrud ut elit dolor. Incididunt voluptate esse et est labore cillum proident duis.
}

\rmk{
    Some remark.
}

\rmkb{
    Some more remark.
}

\section{Pictures}

\begin{figure}[H]
    \center
    \includegraphics[scale=0.1]{img/loo.jpg}
    \caption{Waterloo, ON}
\end{figure}


\chapter{Options}
\section{Introduction to Options (Lecture 21)}
\defn{Derivative}{
    A product that you can issue and sell, however payout is not pre-agreed upon
    that depends on the value of another underlying asset
}
\defn{European Call Option}{
    When issued, what is specified is:
    \begin{itemize}
        \item An underlying asset with value, let's say, $S_t$ at any
        time $t \geq 0$
        \item A future time $T$, called maturity
        \item A price $K$ called "strike"
    \end{itemize}
}

\subsection{European Call Option}
The option is sold after issuance, and the person who holds it has the
\textit{right} to but at time $T$,  the underlying asset from the issuer of the
option, at the price $K$. Thus if $S_T > K$, exercising the right of the option
saves $S_T - K$ dollars.

On the other hand, unlike a long forward position the option does not go with
obligation to make the above purchase, so if $S_T \leq K$ a rational holder
of the option will simply not exercise. In this case the option payout is $0$.
Thus, in any case, payoff at time $T$ is:
$$
(S_T-K)^+ = \max(S_T-K, 0)
$$
\subsection{European Put Option}
Works like European Call option, but holder has right to \textit{sell} the underlying
asset to the issuer for $K$ dollars. Thus they save $K-S_T$ dollars if $K>S_T$ and don't
save or lose anything otherwise. The payoff at time $T$ is
$$
(K-S_T)^+ = \max(K-S_T, 0)
$$
\subsection*{Notation}
\begin{itemize}
    \item $C_e(t,T,K)$: the value at time $t \leq T$ of European Call Option with
    maturity $T$ and strike $K$
    \item $P_e(t, T, K)$ the value at time $t \leq T$ of European Put Option with maturity
    $T$ and strike $K$
\end{itemize}
so clearly for $t=T$:
$$
    C_e(t,T,K) = (S_T-K)^+
$$
$$
    P_e(t,T,K) = (K-S_T)^+
$$

We want to be able to compute the prices of these options for any time $t \leq T$, but this
requires a certain model for now the price $S_t$ of the underlying asset evolves with time $t$. However, some properties can be obtained even without a model. The first being

$$
    (S_T-K)^+ - (K-S_T)^+ = S_T - K
$$
which is equivalently:
$$
    C_e(T,T,K) - P_e(T, T, K) = S_T - K
$$
which can be extended to \textit{any} time $t \leq T$ as:
$$
    C_e(t,T,K) - P_e(t, T, K) = S_T - K \cdot B(t, T)
$$
{\color{red} This is the Put-Call Parity}

\lem{Higher payoff Higher price Lemma}{
Consider 2 products with values $V_1(t)$ and $V_2(t)$ at time $t$ respectively.
We say that if $V_1(T) > V_2(T)$ for any $\omega \in \Omega$ then $V_1(t) > V_2(t)$
at any previous time $t \leq T$
\pf{
    Suppose $V_1(T) > V_2(T)$ for all $\omega \in \Omega$ but $V_1(t) \leq V_2(t)$ (so $V_1(t)$ is undervalued at time $t$ and $V_2(t)$ is overvalued)\\
    Then at time $t$:
        \begin{itemize}
            \item Short sell product 2 to earn $V_2(t)$ dollars
            \item Buy product 1 by paying $V_1(t)$ dollars
            \item Invest the difference in bonds
        \end{itemize}
    \newline
    At time $T$:
        \begin{itemize}
            \item Earn $V_1(t)$
            \item Pay $V_2(t)$ to buy product 2 and deliver it to close short selling position
            \item Receive $\frac{V_2(t) - V_1(t)}{B(t, T)}$ dollars from bonds
        \end{itemize}
    \newline
    So our balance is
    $$
        V_1(T) - V_2(T) + \frac{V_2(t) - V_1(t)}{B(t,T)}  > 0
    $$
    for all $\omega$, which is arbitrage! Contradiction!
}
}
\cor{
   If $V_1(T) = V_2(T)$ for all $\omega \in \Omega$ we can apply above lemma twice to get
   $V_1(t) = V_2(t)$ for all $t \leq T$ and for all $\omega \in \Omega$
}

\thmr{Put-Call Parity}{}{
  $$ C_e(t,T,K) - P_e(t, T, K) = S_T - K \cdot B(t, T) $$
\pf{
    We showed that
    $$
        C_e(t,T,K) - P_e(t, T, K) = S_T - K
    $$
    where the right side is the value of a long forward position at delivery $T$, which equals the payoff
    So we apply the previous result for the products:
    \begin{enumerate}
        \item A portfolio with 1 call and -1 Put, both with strike $K$ and maturity $T$
        \item A long forward position with forward price $K$ and delivery $T$
    \end{enumerate}
    so the above equality is $V_1(T) = V_2(T)$ so by the previous corollary $V_1(t) = V_2(t)$ or in other words
    $C_e(t,T,K) - P_e(t,T,K) = S_t - K\cdot B(t,T)$
}
}
\section{Properties of European and American Options (Lecture 22)}
Last week we proved the "higher payoff-higher price" lemma and we used it to prove
 the Put-Call Parity:

 $$
 C_\euro(t, T, K) - P_\euro(t, T, K)  = S_t - K \cdot B(t,T)
 $$
 where the right hand side is the long forward position at time $t$, which agrees with
 $(F(t,T) -F(t_0, T)) \cdot B(t,T)$ which we already knew when forward contract is made
 at time $t_0 \leq t$. To verify that, recall then $F(t,T) = \frac{S_t}{B(t,T)}$ and
 $F(t_0, T) = K$: forward price used in the contract of this forward position.
 We can also prove that
 $$
    C_\euro(t,T,K) < S_t
 $$
 Indeed, for $t=T$ this reduces to
 $$
    C_\euro(T,T,K) = (S_T - K)^+ < (S_T)^+ = S_T
 $$
 because $f(x) = x^+$ is increasing.
 Then we just apply the "higher payoff-higher price" for one product being the
 stock and the other the call option to get the same result for any $t \leq T$:
 $$
 C_e(t,T,K) < S_t
 $$
 Another inequality that holds is
 $$
 C_e(t,T,K) \geq (S_t - K\cdot B(t,T))^+
 $$
\pf{
By the Put-Call parity we know that
\begin{align*}
  C_e(t,T,K) &- P_e(t,T,K)  = S_t - K\cdot B(t,T)   \\
  C_e(t,T,K) &= S_t - K\cdot B(t,T) + P_e(t,T,K)  \\
  &\geq S_t - K \cdot B(t, T)
\end{align*}
since also $C_\euro(t, T, K) \geq 0$ it follows that
\begin{align*}
    C_\euro(t,T,K) &\geq \max\{0, S_t - K \cdot B(t, T)\} \\
    &= (S_t - K\cdot B(t,T))^{+}
\end{align*}
}

\defn{American Option}{
    An American Option is like an European Option but its holder can
    exercise it at ANY time before maturity $T$
}
\rmkb{
$C_A(t, T, K)$: American Call Option at time $t \leq T$, when $K$ is the strike and $T$ is
the maturity.
}

Properties similar to those of $P_e(t, T, K)$ and $C_e(t, T, K)$ hold for $P_A(t, T, K)$ and
$C_A(t, T, K)$, but the "higher payoff-higher price" lemma cannot be used to prove those,
since the payoffs of Americans are not certainly paid at time $T$. Thus we prove such
properties by the standard no-arbitrage arguments

\ex{
    We have something similar to the Put-Call Parity:
    $$
    S_t - K\cdot B(t,T) \geq C_A(t,T,K)-P_A(t,T,K) \geq S_t - K
    $$
}

Let's prove the above statement in the example
\pf{
We only prove the right part
    $$C_A(t,T,K)-P_A(t,T,K) \geq S_t - K$$
    First we rearrange
    $$C_A(t,T,K)+ K \geq S_t + P_A(t,T,K) $$
    Suppose for the sake of contradiction we have
    $$ C_A(t,T,K) + K < S_t + P_A(t,T,K) $$
    $C_A(t,T,K)$ is American Call Price at time $t$, $K$ is the value at time $t$ of
    $\frac{K}{B(t,T)}$ units of bond that pays 1 dollar at time $T$, each worth $B(t,T)$.
    $P_A(t,T,K)$ American Put Price as time $t$ and $S_t$ is the stock price at time $t$.
    $C_A$ and $K$ are on the small side, so undervalued and the other products are on the big
    side so they are overvalued. Now we just follow the standard procedure. \\
    \underline{At time $t$}
    \begin{itemize}
        \item Issue and sell an American Put to get $P_A(t,T,K)$ dollars
        \item Short sell the stock to get $S_t$ dollars
        \item Buy an American Call by paying $C_A(t,T,K)$
        \item Buy $\frac{K}{B(t,T)}$ units of bond each for $B(t,T)$ dollars, paying $K$
        dollars in total
    \end{itemize}
    \underline{Our Balance}
    $$
    M = P_A(t,T,K) + S_t - (C_A(t,T,K) + K) > 0
    $$
    \underline{Optimal action}: Invest $M$ in bonds, by buying $\frac{M}{B(t,T)}$ units,
    each worth $B(t,T)$ dollars so we pay $M$ dollars. So now our balance becomes 0. \\
    Normally we would go to time $T$, but there is a possibility that the American
    option we sold is exercised before that!
    \underline{Consider 2 cases:}\\
    \underline{Case 1:} American Put we shold is exercised at some $t_0 \in [t, T]$. Then we
    go to time $t_0$.
    \underline{At time $t_0$}: We have the obligation to buy from the option holder,
    the stock for $K$ dollars. Thus, borrow $K$ dollars by issuing and selling
    $\frac{K}{B(t_0, T)}$ that pays 1 dollar at time $T$, each worth $B(t_0, T)$ since we
    are now at time $t_0$, so we receive $K$ dollars. Then, give those dollars to option
    holder who exercised it to get the stock. \\
    \underline{At time T}:
    \begin{itemize}
        \item
            We have a stock received at time $t_0$ but also an open
            short-selling position on the stock at time $t$. Thus we just deliver the stock to close
            the position.
        \item From bonds bought at time $t$, earn
        $$
        \frac{M}{B(t,T)} + \frac{K}{B(t,T)} \text{ dollars}
        $$
        \item From bonds sold at time $t_0$ we must pay $\frac{K}{B(t_0, T)}$ dollars
    \end{itemize}

    Nothing else happens so check balance:
    $$
        \frac{M}{B(t,T)} + \frac{K}{B(t,T)} - \frac{K}{B(t_0, T)}
    $$

    $$
        \frac{M}{B(t,T)} + K \left( \frac{1}{B(t,T)} - \frac{1}{B(t_0, T)} \right)
    $$

    where $\frac{1}{B(t,T)} - \frac{1}{B(t_0, T)} \geq 0$ because
        $B(t,T) = (1+r_e)^{t-T} \leq (1+r_e)^{t_0-T} = B(t_0, T)$ since $t_0 \geq t$



    \underline{Case 2} The American Option we sold is never exercised. Thus we go
    directly to time $T$ \\
    \underline{At time T}: \\
    Again, we must clos short-selling made at time t, but now we do not posses a stock in
    this second case. Thus we exercise the American call we hold (which we didn't have
    to do in case 1) to get the stock for $K$ dollars, and we deliver the stock to close
    short-selling. Then, we also earn from bonds bought at time $t$:
    $$
        \frac{M}{B(t,T)} + \frac{K}{B(t,T)} \text{ dollars}
    $$
    So our balance is
    $$
        \frac{M}{B(t,T)} + \frac{K}{B(t,T)} - K \text{ dollars}
    $$
    $$
        \frac{M}{B(t,T)} + K\left(\frac{1}{B(t,T)} - 1\right)  > 0
    $$
    because $B(t T) = (1+r_e)^{t-T} < 1$ since $t<T$. Thus started with 0 money and ended
    up with positive balance which is arbitrage $\rightarrow$ Contradiction.
}

\thmr{}{}{
    Consider 2 options $A$, $B$ with prices $A(t)$ and $B(t)$ respectively at any time $t$.
    If the holder of $A$ can also use it exactly as $B$, then $A(t) \geq B(t)$ for all $t$
}
Sketch of proof. Suppose $A(t) < B(t)$. So we issue and sell $B$, buy $A$ and invest the
positive balance $B(t) - A(t)$ in bonds (all at time $t$). Now when $B$ is exercised, we
exercise $A$ using it exactly as $B$, so we earn from $A$ exactly what we must pay to the one
who exercises $B$. Thus we are only left with our investments in bonds, which leads to
arbitrage. \\
\\
Consequences of the above theorem
\begin{enumerate}
    \item $C_A(t, T, K) \geq C_e(t, T, K)$ and $P_A(t,T,K) \geq P_e(t, T, K)$ because an
    American Option can be used exactly as the corresponding European Option (just by deciding
    to exercise at time $T$)
    \item American Options with more distant maturities have higher prices
    $C_A(t, T, K) \geq C_A(t, T', K)$ and $P_A(t,T,K) \geq P_A(t, T', K)$ whenever
    $T \geq T'$ because an american with maturity T can be used as the
\end{enumerate}


\section{Option Results \& Binomial Model}
While we have
$$
C_A(t, T, K) \geq C_e(t, T, K)
$$
when the stock pays no dividends we actually have
$$
C_A(t, T, K) = C_e(t, T, K)
$$
and it is optimal for an investor to exercise the American at time $T$

We have obtained various identities and bounds for option prices, but to explicitly
compute those prices we need to know how the price $S_t$ of the underlying stock
evolves with time $t$.

We focus on $2$ models:
\begin{enumerate}
    \item The Binomial model (Discrete time)
    \item The Black-Scholes model (Continuous time)
\end{enumerate}

\subsection{The Binomial Model}
Given a time period $[0, T]$, a stock $S$ with price $S_t$ at time $t \in [0.T]$, and
a bond that pays 1 dollar at time $T$ with price $B(t, T)$ at time $t \in [0.T]$

Everything (changes in the stock price and interest compounding which changes bond
prices) happen at the following times: \\

$t = 0, t = h, t = 2h, \cdots t = Nh = T$ i.e prices change every $h$ units of time
and there are $N = \frac{T}{h}$ steps in total. Let $S(n) = S_{nh}$ (stock price
at the n-th step). \\

\underline{Stock return:}
$$
    R(n+1) = \frac{S(n+1) - S(n)}{S(n)}
$$

Under the binomial model, the returns at different steps are independent random
variables, with:

$R(n)=

U & \text{with probability } p\\
D  & \text{with probability } 1-p
\end{cases}$

for all $n \in [1, 2, \cdots, N]$, where: $-1 < D < U$ \\

using the equality for returns the previous equality becomes:

$S(n+1)=
\begin{cases}
S(n)(1+U) & \text{with probability } p\\
S(n)(1+D)  & \text{with probability } 1-p
\end{cases}$

for $n \in \{0, 1, \cdots N-1 \}$

We can now compute the price:
\begin{align*}
    S(n) &= S(n-1) \cdot (1 + R(n)) \\
        &= S(n-2) \cdot(1 + R(n-1)) \cdot (1 + R(n)) \\
        &\hspace{5em}\vdots\\
        &= S(0) \cdot \prod_{i=1}^n(1+R(i)) \\
        &= S(0) \cdot (1 + U)^k \cdot (1 + D)^{n-k}
\end{align*}
where $k$ is the number of the returns $R(i)$ that equal $U$, so
remaining $n-k$ are equal to $D$.

\underline{Notation:} If we observe the returns $R(1), R(2), \cdots, R(n)$ until
the $n$-th step, we write:
$S^{R(1)R(2) \cdots R(n)}(n)$ for the price of the stock at the $n$-th step under
those observed returns. \\
\ex{
    $S^{UUD}(3)$ the price of the stock at the 3rd step when $R(1) = U$,
    $R(2) = U$, and $R(3) = D$ which is equal $S(0) \cdot (1+U)^2 \cdot (1+D)$
}

If $\Omega$ is the underlying sample space, each evolution scenario for the market
scenario for the market corresponds to a $\omega \in \Omega$, which can be identified
with the sequence of returns $R(i)(\omega)$ under that $\omega$
$$
\omega = R(1)(\omega)R(2)(\omega) \cdots R(N)(\omega)
$$

Consider now one possible price at step $n$:
$$
    S(0)(1+U)^k \cdot (1+D)^{n-k}
$$
The probability of the price at time $n$ being exactly the above for fixed
$n$ and $k$ is precisely:
$$
    \Pr(S(n) = S(0)(1+U)^k \cdot (1+D)^{n-k}) = \binom{n}{k}p^k(1-p)^{n-k}
$$

\underline{Bond Prices:} \\
Consider a bond that pays $1$ dollar at step $m$ (time $m \cdot h$). Its price
at step $n \leq m$ (time $n\cdot h$) is then
\begin{align*}
    B(nh, mh) &= (1+r_e)^{-(mh-nh)} \\
              &= ((1+r_e)^h)^{(n-m)}
\end{align*}
so that $B(nh, mh) = (1+r_h)^{n-m}$. For a single step ($m=n+1$), $B(nh, (n+1)h) =
\frac{1}{1+r_n}$. The bond's return in $[nh, (n+1)h]$ is then:
$$
    \frac{B((n+1)h, (n+1)h - B(nh, (n+1)h}{B(nh, n(+1)h} = r_n \rightarrow \text{interest rate b/w consecutive steps}
$$
\thmr{}{}{
    Under the Binomial model, to have no arbitrage we require:
    $$D < r_h < U$$
    between 2 consecutive steps, the bond's return must be strictly between the
    2 possible returns of the stock.
    \pf{
        We prove $r_h < U$. Suppose for contradiction $r_h \geq U$ so that
        $r_h \geq U > D$ this means that the bond has for sure
        a better return compared to the stock, so we buy it by
        short selling the stock. \\
        \underline{At time 0}: Short sell the stock to earn $S(0)$ which we
        invest in buying bonds that pay $1$ dollar at the next step, each bond
        is worth $B(0,h)$, so we by $\frac{S(0)}{B(0,h)}$ units. \\
        \underline{At time h (step 1)}: We earn $\frac{S(0)}{B(0,h)} = S(0)
        \cdot(1+r_h)$ dollars. We must close short selling, so we pay $S(1)$ to
        buy the stock and we deliver it to close short selling. Our balance
        at this point is now
        \begin{align*}
            S(0) \cdot (1+r_h) - S(1)  &= S(0)(1+r_h) - S(0)(1+R(1)) \\
            &= S(0)(r_h-R(1)) \\
            &= S(0)
                \begin{cases}
                    r_h - U & \text{with probability } p\\
                    r_h - D  & \text{with probability } 1-p
                \end{cases}
        \end{align*}
        which is certainly $\geq 0 $ since $r_h \geq U > D$ and with
        \underline{positive probability $1-p > 0$} it is a strictly
        positive profit of $r_h - D > 0$. This is arbitrage, thus we have
        reached a contradiction.
    }
}



\section{Lecture 24}
\subsection{Self Financing Property}
Suppose at the $n$-th step we hold $\alpha_S(n)$ units of stock
(each unit workh $S(n)$) and $\alpha_B(n)$ units of bond that pays
1 dollar at step $N$ (each unit worth $B(nh, T)$), so our portfolio
has value
\begin{equation}
    V(n) = \alpha_S(n)S(n) + \alpha_B(n)B(nh, T)
\end{equation}

At the next step, values of products change the stock $S(n) \longrightarrow S(n+1)$
and the bond $B(nh, T) \longrightarrow B((n+1)h, T) = B(nh,T)(1+r_n)$ so our
portfolio value becomes

\begin{equation}
    V(n+1) = \alpha_S(n)S(n+1) + \alpha_B(n)B((n+1)h, T)
\end{equation}

Then, we can rebalance: sell stocks and use money earned to buy bonds
or the opposite, so amounts of assets we hold change:
$\alpha_S(n) \rightarrow \alpha_S(n+1)$ and
$\alpha_B(n) \rightarrow \alpha_B(n+1)$
However if no extra money is added no money is put aside
the value of the portfolio remains the same after $\alpha_S(n)$ and
$\alpha_B(n)$ change ({\color{red} self-financing property})
\begin{equation}
    V(n+1) = \alpha_S(n+1)S(n+1) + \alpha_B(n+1)B((n+1)h, T)
\end{equation}

Subtracting 2.3 from 2.2 we get the {\color{red} Self-financing property}
\begin{equation}
    0 = (\alpha_S(n+1) - \alpha_S(n))\cdot S(n+1) + (\alpha_B(n+1) - \alpha_B(n))\cdot B((n+1)h, T)
\end{equation}

Alternatively we can subtract 2.2 from 2.1 to find
\begin{equation*}
    V(n+1) - V(n) = \alpha_S(n) \cdot (S(n+1) - S(n)) + \alpha_B(n) \cdot (B((n+1)h, T) - B(nh, T))
\end{equation*}
\subsection{Hedging and Pricing}
Suppose that aprt from bond and stock we have an option, whose value at step
$n+1$ is:
\begin{equation*}
    P(n+1) = H_{n+1}(S(n+1))
\end{equation*}
e.g if the option is a European Call maturing at step $n+1$: $H_{n+1}(x) = (x-K)^+$
\subsubsection{How to find option price $P(n)$ at the previous step n?}
The idea is to construct a portfolio at step $n$, whose price at step $n+1$ will always
coincide with the option price at step $n+1$, by approximately selecting the $\alpha_S(n)$ and
$\alpha_B(n)$ that do the work. There are two scenarios
\begin{enumerate}[(a)]
    \item with probability $p$ \\
    the portfolio value becomes
        \begin{align*}
            V(n+1) &= \alpha_S(n) S(n+1) + \alpha_B(n) B((n+1)h, T) \\
                &= \alpha_S(n) S(n)(1+U) + \alpha_B(n) B((n+1)h, T)
        \end{align*}
        the option value becomes
        \begin{equation*}
            P(n+1) = H_{n+1}(S(n+1)) = H_{n+1}(S(n)(1+U))
        \end{equation*}
    \item with probability $1-p$ \\
    the portfolio value becomes
        \begin{align*}
            V(n+1) &= \alpha_S(n) S(n+1) + \alpha_B(n) B((n+1)h, T) \\
                &= \alpha_S(n) S(n)(1+D) + \alpha_B(n) B((n+1)h, T)
        \end{align*}
        the option value becomes
        \begin{equation*}
            P(n+1) = H_{n+1}(S(n+1)) = H_{n+1}(S(n)(1+D))
        \end{equation*}
\end{enumerate}

We want the portfolio value to coincide with the option value at step $n+1$:
$$
    V(n+1) = P(n+1)
$$
under both scenarios, so that by the "Higher payoff- higher price" lemma we will then
have equality of values at step $n$
$$P(n)=V(n)$$
Requiring that $V(n+1) = P(n+1)$ under both scenarios gives a 2x2 system which solving
yields
$$
    \alpha_S(N) = \frac{1}{S(n)} \cdot \frac{H_{n+1}(S(n)(1+U))-H_{n+1}(S(n)(1+D))}{U-D}
$$
and
$$
    \alpha_B(N) = \frac{1}{S(n)} \cdot \frac{H_{n+1}(S(n)(1+U))-H_{n+1}(S(n)(1+D))}{(D-U)(1+r_n)^{n+1-N}}
$$
as expected those only depend on $S(n)$ and deterministic quantities, so they are known at step
$n$. Now, by what we said above, our option price at step $n$ is
$$
P(n) = V(n) = \alpha_S(n)S(n) + \alpha_B(n)B(nh, T)
$$
where we can plug the $\alpha_S(n)$, and $\alpha_B(n)$ we computed to find:
$P(n) = \cdots = H_n(S(n))$

so again it is a function of the stock price at the step at which the option price is
computed. However at this step $n$:
\begin{equation*}
    H_n(x) = \frac{1}{1+r_n} \left(\frac{U-r_n}{U-D}H_{n+1}((1+D)x) + \frac{r_n - D}{U-D}H_{n+1}
    ((1+U)x)\right)
\end{equation*}

which differs from the $H_{n+1}(x)$ that we have the next step $n+1$

Now we write:

$$
P(n) = \frac{1}{1+r_n}\mathbb{E}^{p^*}(H_{n+1}(S(n+1)) |\mathcal{F}_n)
$$

where expectation $\mathbb{E}^{p*}$ is taken in an "artificial" regime where:


$S(n+1)=\begin{cases}
S(n)(1+U) & \text{with probability } p^* = \frac{r_n-D}{U-D} \\
S(n)(1+D)  & \text{with probability } 1-p^*
\end{cases}$


Thus we deduce that we can rewrite the above as
$$
P(n) = \frac{1}{1+r_n}\mathbb{E}^{p^*}(P(n+1)|\mathcal{F}_n)
$$

Dividing now by $(1_{r_n})$ and defining $\tilde{P}(k) = \frac{P(k)}{(1+r_n)^k}$ we get
which is the price of the option at time $k$ discounter to time $0$ where the discount
factor is $(1+r_n)^k = B(0, kh)$:
we get from the last equality:
\begin{equation*}
    \tilde{P}(n) = \mathbb{E}^{p^*}(\tilde{P}(n+1)|\mathcal{F}_n)
\end{equation*}

{\color{red} This is the martingale property of the discounted price $\tilde{P}(n)$}
and because we also have:
$$
\tilde{P}(n-1) = \mathbb{E}^{p^*}(\tilde{P}(n)|\mathcal{F}_{n-1})
$$

we can get:

$$
    \tilde{P}(n-1) = \mathbb{E}^{p^*}( \mathbb{E}^{p^*}(\tilde{P}(n+1)|\mathcal{F}_n)
 | \mathcal{F}_{n-1})
$$

notice here we are conditioning on $\mathcal{F}_{n-1}$ plus things observed at step $n$
so by the tower rule the inner expectation disappears and we get:

$$
    \tilde{P}(n-1) = \mathbb{E}^{p^*}(\tilde{P}(n+1)|\mathcal{F}_{n-1})
$$

which can be extended to any step $k$:

$$
    \tilde{P}(n-k) = \mathbb{E}^{p^*}(\tilde{P}(n+1)|\mathcal{F}_{n-k})
$$

for any $k \in \{0, 1, \cdots, n\}$ giving the option price
at ANY step before $n+1$

\section{Lecture 25}
skip for now
\section{Lecture 26}
Went over but lots of diagrams so lowkey dont want to type it up
\section{Brownian motion and Ito's calculus (Lecture 27)}
In Continuous time models we will only cover the Black-Scholes model. In this lecture we develop
the necessary mathematical tools.

Start with a probability space

$$
    (\Omega, \mathcal{F}, \mathbb{P})
$$
\defn{Continuous Time Stochastic Process}{
    A continuous time stochastic process is a function $X$ that maps any $(t,\omega)$
    for $t \geq 0$ and $\omega \in \Omega$ to the value $X(t,\omega)$ which describes the value
    of a randomly evolving quantity (e.g a stock price) at time $t$ under scenario $\omega$
}
\begin{itemize}
    \item For a fixed outcome $\omega$: $X(\cdot, \w)$ is a function of the time variable $t$, which maps any
        time $t$ to the value $X(t, \w)$ (describes the evolution of the process with time, under a fixed
        scenario/outcome $\w$)
    \item For a time $t$: $X(t, \cdot)$ is a random variable that describes the possible values of the
        process at a fixed time $t$, under different scenarios/outcomes $\w$
\end{itemize}
\defn{Brownian Motion}{
    A stochastic process $W$ that maps any time $t$ and outcome $\w$ to $W(t,\w) = W_t$ is called a
    Brownian motion when:
    \begin{enumerate}
        \item $W_0=W(0,\w) = 0$
        \item The function $W(\cdot, \w)$, which is a function of $t$, is a \textit{continuous function} of
        $t$ under any possible scenario $\w$
        \item For $t_1 \leq t_2 \leq \cdots \leq t_n$, the increments $W(t_n, \cdot) - W(t_{n-1},\cdot)$,
        $W(t_{n-1}, \cdot) - W(t_{n-2},\cdot)$, $\cdots$, $W(t_2, \cdot) - W(t_1, \cdot)$ (increments over
        non-lapping intervals) are \textit{independent} random variables.
        \item For $t \geq s$
            $$
                W(t, \cdot)-W(s, \cdot) \sim N(0, t-s)
            $$
    \end{enumerate}
}
\subsection*{Calculus of Brownian Motion}
Let's try to compute the differential
$$
    dW_t = W_{t +dt} - W_t
$$
\renewcommand{\Var}{\text{Var}}
\renewcommand{\E}{\mathbb{E}}
By property 4 of Brownian Motion $dW_t \sim N(0,dt)$ so:
\begin{itemize}
    \item $\E(dW_t) = 0$, $\Var(dW_t)=dt$
    \item $\E((dW_t)^2) = \Var(dW_t) + \E^2(dW_t)  = dt$
\end{itemize}
Moreover $\Var((dW_t)^2)$ is proportional to $(dt)^2$ and thus negligible so:
$$
    dW_t^2 = \E((dW_t)^2) = dt
$$
\subsection*{Recap of ordinary calculus}
    Consider $z = f(x,y)$, we want to compute $dz$, a small change in $z$ that occurs
    from a small change $dx$ in $x$ and a small change $dy$ in $y$
    $$
        dz = df(x,y) = f(x+dx, y+dy) - f(x,y)
    $$
    By taylor expanding we get
    $$
        f(x+dx, y+dy) = f(x, y) + f_x(x,y)dx + f_y(x,y)dy + \frac{1}{2}f_{xx}(x,y)(dx)^2 + \frac{1}{2}f_{yy}(x,y)(dy)^2 + f_{xy}(x,y)dx dy
    $$
    so we can easily see that
    $$
        dz = f_x(x,y)dx + f_y(x,y)dy + \frac{1}{2}f_{xx}(x,y)(dx)^2 + \frac{1}{2}f_{yy}(x,y)(dy)^2 + f_{xy}(x,y)dx dy
    $$
    Assume now that $x=x(t)$ and $y=y(t)$, so small changes in those occure from a small chnage dt in $t$. Then:
    \begin{align*}
        dz = f_x(x(t),y(t))dx(t) + f_y(x(t),y(t))dy(t) + \frac{1}{2}f_{xx}(x(t),y(t))(dx(t))^2 &+ \frac{1}{2}f_{yy}(x(t),y(t))(dy(t))^2\\ &+ f_{xy}(x(t),y(t))dx(t)dy(t)
\   \end{align*}
    Now if $x(t)$, $y(t)$ are differntiable in $t$: $dx(t) = x'(t)dt + O((dt)^2)$ and $dy(t) = y'(t)dt + O((dt)^2)$ which
    can be plugged into the above equation and we can observe that all terms that are accompanied by a factor of $(dt)^2$,
    $(dt)^3$ or higher are regarded as $O$ because $(dt)^n$ for $n>1$ is much smaller than $dt$. Then it is easy to see that
    ALL terms involving 2nd order derivativees will vanish in the equation and it becomes
    $$
        dz = f_x(x(t),y(t))x'(t)dt + f_y(x(t),y(t))y'(t)dt
    $$
    which is just the standard chain rule of ordinary calculus. \\

    Now we can consider if $x(t) = t$ but $y(t) = W_t$ is a Brownian motion (non-differentiable). We cannot plug into
    $dy(t) = y'(t)dt + O((dt)^2)$ because $y'(t)$ does not exist. Actually the double derivative with resepect to $y$
    term becomes
    $$
    \frac{1}{2}f_{yy}(x(t),y(t))(dy(t))^2 = \frac{1}{2}f_{yy}(t, W_t)(dW_t)^2 = \frac{1}{2}f_{yy}(t, W_t)dt
    $$

    \rmkb{
        other $2$nd order derivatives will still vanish as the $f_{xy}$ term goes with $dW_tdt$ which is also very much
        smaller than $dt$ and thus is regarded as 0
    }

    So our differential becomes

    \defn{Itô's Formula}{
        $$
            dz = df(t, W_t) = f_x(t, W_t)dt + f_y(t, W_t)dW_t + \frac{1}{2}f_{yy}(t, W_t)dt
        $$
    }

    If we integrate the above in $[s,t]$
    \begin{align*}
        f(t, W_t) - f(s, W_s) &= \int_s^t df(r, W_r) \\
                              &= \int_s^t f_x(r, W_r)dr + \int_s^t \frac{1}{2}f_{yy}(r, W_r)dr + {\color{purple}\int_s^t f_y(r, W_r)dW_r} \\
    \end{align*}
    where the first two integrals are ordinary integrals and the last one is an {\color{purple}Itô integral}
    \subsection*{Properties of {\color{purple}Itô integrals}}
    Lets compute
    $$
    \E\left[\int_a^b f(r)dW_r\right] = \int_a^b \E[f(r)dW_r]
    $$
    so if we assume that the values of $f$ until any time $s$ are \underline{known} provided that the values of $W$ until
    that same time are also known then
    $$
    \E[f(r)dW_r] = \E[f(r)(W_{r+dr} - W_r)]
    $$
    where $W_{r+dr} - W_r$ is independent of $W_r^'$ for $r' \leq r$ and because the latter fully determine the value of
    $f(r)$, $W_{r+dr} - W_r$ is independent of $f(r)$

    \begin{align*}
        \E[f(r)(W_{r+dr} - W_r)] &= \E[f(r)]\E[W_{r+dr} - W_r] \\
                                 &= \E[f(r)]\cdot \E[N(0, dr)] \\
                                 &= \E[f(r)]\cdot 0 \\
                                 &= 0
    \end{align*}
    \break
    Thus we have that $\E\left[\int_a^b f(r)dW_r\right] = 0$ which extends to $$\E\left[\int_s^t f(r)dW_r | \mathcal{F}_s\right] = 0$$
    where $\mathcal{F}_s$ represents everything observable until time $s$ and from that we can obtain the following theorem

    \thmr{Martingale Property of the Ito Integral}{}{
        Provided that $f$ is $W$-adapted (has known values any time until which the Brownian motion is observed)
        \begin{align*}
            \E\left[\int_0^t f(r)dW_r | \mathcal{F}_s\right] &= \E\left[\int_s^t f(r)dW_r + \int_0^s f(r)dW_r \vert \mathcal{F}_s\right] \\
                                                             &= \int_0^s f(r)dW_r
        \end{align*}
    }
    The first integral was shown to have 0 expectation above and the second one is known given $\mathcal{F}_s$.
    Compute $\E\left[W_3^2 \mid W_1 = 5\right]$
    We need to express $W_t^2$ as a combination of integrals and use the above properties of expecations. We use
    Ito's formula on $f(x,y) = y^2$ so that $f(t,W_t) = W_t^2$. $f_x(x,y) = 0$, $f_y(x,y) = 2y$ and $f_{yy}(x,y) = 2$ so
    by Itô's formula (integral form):
    \begin{align*}
        f(t, W_t) - f(s, W_s) &= \int_s^t f_x(r, W_r)dr + \int_s^t f_y(r, W_r)dW_r + \frac{1}{2}\int_s^t f_{yy}(r, W_r)dr \\
                              &= 0 + \int_s^t 2W_r dW_r + \int_s^t 1dr \\
                              &= \int_s^t 2W_r dW_r + (t-s)
    \end{align*}
    then taking expectations given $\mathcal{F}_s$ we get
    \begin{align*}
        \E[W_t^2 \mid \mathcal{F}_s] - \E[W_s^2 \mid \mathcal{F}_s] &= \E\left[\int_s^t 2W_r dW_r \mid \mathcal{F}_s\right] + (t-s) \\
                                                              &= (t-s) \\
                                                              &\vdots \\
        \E[W_t^2 \mid \mathcal{F}_s] &= \E[W_s^2 \mid \mathcal{F}_s] + (t-s) \\
                                  &= W_s^2 + (t-s) \\
    \end{align*}
    \renewcommand{\F}{\mathcal{F}}
    where the expectation of the first integral is 0 by property of the Itô integral and the expecation of $W_s^2$ given $\F_s$ is
    just $W_s^2$ because it is known and since $\E[W_3^2 \mid W_1 = 5] = \E[W_2^2 \mid \F_1]$ when $W_1 = 5$ is observed, all we need is
    to plug in $t=3$, $s=1$ and $W_1 = 5$ into the above equation to get what we need
    $$
        \E[W_3^2 \mid W_1 = 5] = 5^2 + (3-1) = 25 + 2 = 27
    $$


\end{document}

