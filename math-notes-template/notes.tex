\documentclass[oneside]{book}

\usepackage{amsmath, amsthm, amssymb, amsfonts}
\usepackage{thmtools}
\usepackage{graphicx}
\usepackage{setspace}
\usepackage{geometry}
\usepackage{float}
\usepackage{hyperref}
\usepackage[utf8]{inputenc}
\usepackage[english]{babel}
\usepackage{framed}
\usepackage[dvipsnames]{xcolor}
\usepackage{environ}
\usepackage{tcolorbox}
\tcbuselibrary{theorems,skins,breakable}
\usepackage{eurosym}
\usepackage{enumerate}

\setstretch{1.2}
\geometry{
    textheight=9in,
    textwidth=5.5in,
    top=1in,
    headheight=12pt,
    headsep=25pt,
    footskip=30pt
}

% Variables
\def\notetitle{MATH 423}
\def\noteauthor{
    \textbf{Omer Junedi} \\
    University of Michigan
    }
\def\notedate{Winter 2025}

% The theorem system and user-defined commands
\input{theorems.tex}
\input{commands.tex}
\newcommand{\w}{\omega}

% ------------------------------------------------------------------------------

\begin{document}
\title{\textbf{
    \LARGE{\notetitle} \vspace*{10\baselineskip}}
    }
\author{\noteauthor}
\date{\notedate}

\maketitle
\newpage

\tableofcontents
\newpage

% ------------------------------------------------------------------------------

\chapter{Examples}

\section{Theorem System}

\defn{Definition Name}{
    A defintion.
}

\thmr{Theorem Name}{mybigthm}{
    A theorem.
}

\lem{Lemma Name}{
    A lemma.
}

\fact{
    A fact.
}

\cor{
    A corollary.
}

\prop{
    A proposition.
}

\clmp{}{
    A claim.
}{
    A reference to Theorem~\ref{thm:mybigthm}
}

\pf{
    Veniam velit incididunt deserunt est proident consectetur non velit ipsum voluptate nulla quis. Ea ullamco consequat non ad amet cupidatat cupidatat aliquip tempor sint ea nisi elit dolore dolore.

    Laboris labore magna dolore eiusmod ea ex et eiusmod laboris. Et aliquip cupidatat reprehenderit id officia pariatur.
}

\ex{
    Nostrud esse occaecat Lorem dolore laborum exercitation adipisicing eu sint sunt et. Excepteur voluptate consectetur qui ex amet esse sunt ut nostrud qui proident non. Ipsum nostrud ut elit dolor. Incididunt voluptate esse et est labore cillum proident duis.
}

\rmk{
    Some remark.
}

\rmkb{
    Some more remark.
}

\section{Pictures}

\begin{figure}[H]
    \center
    \includegraphics[scale=0.1]{img/loo.jpg}
    \caption{Waterloo, ON}
\end{figure}


\chapter{Options}
\section{Introduction to Options (Lecture 21)}
\defn{Derivative}{
    A product that you can issue and sell, however payout is not pre-agreed upon
    that depends on the value of another underlying asset
}
\defn{European Call Option}{
    When issued, what is specified is:
    \begin{itemize}
        \item An underlying asset with value, let's say, $S_t$ at any
        time $t \geq 0$
        \item A future time $T$, called maturity
        \item A price $K$ called "strike"
    \end{itemize}
}

\subsection{European Call Option}
The option is sold after issuance, and the person who holds it has the
\textit{right} to but at time $T$,  the underlying asset from the issuer of the
option, at the price $K$. Thus if $S_T > K$, exercising the right of the option
saves $S_T - K$ dollars.

On the other hand, unlike a long forward position the option does not go with
obligation to make the above purchase, so if $S_T \leq K$ a rational holder
of the option will simply not exercise. In this case the option payout is $0$.
Thus, in any case, payoff at time $T$ is:
$$
(S_T-K)^+ = \max(S_T-K, 0)
$$
\subsection{European Put Option}
Works like European Call option, but holder has right to \textit{sell} the underlying
asset to the issuer for $K$ dollars. Thus they save $K-S_T$ dollars if $K>S_T$ and don't
save or lose anything otherwise. The payoff at time $T$ is
$$
(K-S_T)^+ = \max(K-S_T, 0)
$$
\subsection*{Notation}
\begin{itemize}
    \item $C_e(t,T,K)$: the value at time $t \leq T$ of European Call Option with
    maturity $T$ and strike $K$
    \item $P_e(t, T, K)$ the value at time $t \leq T$ of European Put Option with maturity
    $T$ and strike $K$
\end{itemize}
so clearly for $t=T$:
$$
    C_e(t,T,K) = (S_T-K)^+
$$
$$
    P_e(t,T,K) = (K-S_T)^+
$$

We want to be able to compute the prices of these options for any time $t \leq T$, but this
requires a certain model for now the price $S_t$ of the underlying asset evolves with time $t$. However, some properties can be obtained even without a model. The first being

$$
    (S_T-K)^+ - (K-S_T)^+ = S_T - K
$$
which is equivalently:
$$
    C_e(T,T,K) - P_e(T, T, K) = S_T - K
$$
which can be extended to \textit{any} time $t \leq T$ as:
$$
    C_e(t,T,K) - P_e(t, T, K) = S_T - K \cdot B(t, T)
$$
{\color{red} This is the Put-Call Parity}

\lem{Higher payoff Higher price Lemma}{
Consider 2 products with values $V_1(t)$ and $V_2(t)$ at time $t$ respectively.
We say that if $V_1(T) > V_2(T)$ for any $\omega \in \Omega$ then $V_1(t) > V_2(t)$
at any previous time $t \leq T$
\pf{
    Suppose $V_1(T) > V_2(T)$ for all $\omega \in \Omega$ but $V_1(t) \leq V_2(t)$ (so $V_1(t)$ is undervalued at time $t$ and $V_2(t)$ is overvalued)\\
    Then at time $t$:
        \begin{itemize}
            \item Short sell product 2 to earn $V_2(t)$ dollars
            \item Buy product 1 by paying $V_1(t)$ dollars
            \item Invest the difference in bonds
        \end{itemize}
    \newline
    At time $T$:
        \begin{itemize}
            \item Earn $V_1(t)$
            \item Pay $V_2(t)$ to buy product 2 and deliver it to close short selling position
            \item Receive $\frac{V_2(t) - V_1(t)}{B(t, T)}$ dollars from bonds
        \end{itemize}
    \newline
    So our balance is
    $$
        V_1(T) - V_2(T) + \frac{V_2(t) - V_1(t)}{B(t,T)}  > 0
    $$
    for all $\omega$, which is arbitrage! Contradiction!
}
}
\cor{
   If $V_1(T) = V_2(T)$ for all $\omega \in \Omega$ we can apply above lemma twice to get
   $V_1(t) = V_2(t)$ for all $t \leq T$ and for all $\omega \in \Omega$
}

\thmr{Put-Call Parity}{}{
  $$ C_e(t,T,K) - P_e(t, T, K) = S_T - K \cdot B(t, T) $$
\pf{
    We showed that
    $$
        C_e(t,T,K) - P_e(t, T, K) = S_T - K
    $$
    where the right side is the value of a long forward position at delivery $T$, which equals the payoff
    So we apply the previous result for the products:
    \begin{enumerate}
        \item A portfolio with 1 call and -1 Put, both with strike $K$ and maturity $T$
        \item A long forward position with forward price $K$ and delivery $T$
    \end{enumerate}
}
